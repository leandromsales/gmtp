\chapter{Considera��es Finais}
\label{cap:conclusao}

TBD

% \section{Perguntas Frequentes}


\section{Conclus�es}

* Por que PS? � invariante a distribui��o do tamanho do fluxo, ou seja, n�o importa ao diferentes s�o os tamanhos dos fluxos na rede. Mesmo com longos fluxos, o compartilhamento do canal � melhor porque estabiliza mais r�pido se comparado com abordagens de monitoramento de perdas.
   RCP: atualiza a taxa (simples e pr�tico): aproxima a taxa de transmiss�o oferecida aos fluxos da taxa do PS, sem o conhecimento do n�mero exato de fluxos; baseia-se no tamanho da fila, no tr�fego agregado de entrada e no RTT m�dio do tr�fego passando atrav�s do link
   R(t) � atualizado de acordo com h0.
   Cada roteador tem que fazer tr�s coisas
       - offer the same rate to every flow
       - preencher a fila de sa�da com os pacotes que chegam na fila de entrada
       - manter a fila de entrada e sa�da pequena
   Long-flow: 100\% link utilization
   Short-flow: reduce flow completion time,
   Flash-crownd no RCP: pode ter perde
   A quest�o �: em vez de definir aumentos ou uma diminui��es incrementais na janela com base no RTT, no RCP a pergunta �: existe uma taxa de transmiss�o que os roteadores podem pedir aos fluxos transmitirem e emular o PS. Qual seria essa taxa? como encontr�-la de forma simples?
   Para ter um FCT pequeno, tem que ter alta utiliza��o do link, ent�o precisa-se garantir  que o buffer n�o estoure (rever 28:30min da defesa da Nandita)
   Nem sempre o XCP � ruim, quando s� tem long-lived, mas em cen�rios din�micos (diferentes tamanhos de fluxos) it performs badly and short lived flows last almost the same time of some long lived flows. Quando aumenta o tamanho m�dio do fluxo para se aproximar ao Bandwidth-delay product, you approach the regime where you have a few high Bandwidth flows multiplex in a large link (51:00min)
   Diferentes topologias: 
     - Sem manter estado por fluxo
     - TCP: perda de pecotes ou atraso
     - XCP: alcan�a desempenho similar ou melhor do que HS-TCP para long-lived flows
            n�o � 100\% fair em determinados momentos
            em fluxos mistos (tamanho) e na din�mica da rede n�o � bom
            multiplica��es e divis�es � sim ruim (Nandita -> Cisco)
            
            
     - CC: na literatura, em geral, s� focam em usar o m�ximo do link


\section{Trabalhos Futuros}
