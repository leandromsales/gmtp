The increasing number of broadband Internet connections results in a greater demand for quality of service from streaming multimedia systems and the rising costs of network bandwidth for the content distributors. The state of the practice works such as Denacast/CoolStreaming, prefer to use P2P/CDN architecture, aiming at disseminating multimedia content efficiently (scalability and quality). The state of the art proposals such as the Content Centric Network (CCN), use pull service model and cache of frequently accessed data on the routers. The problem of practical approaches is the consequent increase in the consumption of network resources, resulting from the strategies adopted in application protocols, proposed to address limitations of the lower layers and arranged without any interoperability. As a consequence, the traffic of duplicate datagrams is increased, limiting the performance of live multimedia distribution systems - the Internet was not designed for this purpose. On the other hand, CCN considers a service model that is not suitable for transport of transient data, delegating the implementation of important services to client nodes, when they should be provided by the network. As a result, it increases the amount of control packets depending on the upstream bandwidth, which is, in general, lower compared to the downstream bandwidth, resulting in low quality multimedia services.

For these and other reasons detailed in this document, we propose the Global Media Transmission Protocol (GMTP), presented in two perspectives: design and performance. In the design perspective, a multi-layer protocol is proposed to efficiently disseminate live media through a network of cooperative routers. This partnership between routers is determined by the streaming servers, based on the bandwidth of network channels explicitly shared by the routers already in use to disseminate content, obtained and updated periodically. In the performance perspective, GMTP was studied by comparing it to the aforementioned proposals, evaluating key metrics of quality of service. Based on the results obtained by simulating a representative scenario,  the proposed GMTP achieved \ut{61.44}{\%} performance improvement  better than Denacast/CoolStreaming and \ut{36.27}{\%} better than the CCN-TV.