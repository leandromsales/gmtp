TBD

% The transport of multimedia content in real time over the Internet is essential
% in applications such as voice over IP (VoIP), video conferencing, games and
% WebTV. In the last years, there has been an increasing number of applications
% with a single transmitting node and thousands of receiving nodes. For example,
% YouTube Live broadcasted games of the American Cup 2011 to 100 million users. A
% critical aspect in these applications is the overhead of using computing
% resources on servers and transmission channels. This has demanded exorbitant
% investment in building Content Delivery Networks (CDNs) by companies like
% Google, Netflix etc. However, the applications are still at the mercy of
% traditional transport protocols (TCP, UDP, SCTP and DCCP), which were not
% designed to transmit multimedia data in a large scale. To address the
% limitations of these protocols, developers implement a mechanism to
% use network resources more efficiently, specially using P2P (Peer to
% Peer) architectures. However, these solutions are palliative because they are
% disseminated in the form of systems or application protocols, where it is
% impossible to avoid scattering and fragmentation of them, increasing the
% complexity of large-scale deployment in the Internet. In this work it is
% proposed a multimedia transport protocol called Global Media Transmission
% Protocol (\mudccp). The \mudccps dynamically builds an overlay network between
% nodes in a P2P fashion, without the direct influence of the application and
% supporting multicast transmission and congestion control. Preliminary results
% indicate that the \mudccps efficiently utilizes network resources and improves
% scalability of multimedia applications, avoiding the rework development by
% concentrating the main mechanisms in a single transport protocol.