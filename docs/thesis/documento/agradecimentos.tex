Em primeiro lugar, agrade�o a Deus e a Jesus Cristo por ter me proporcionado maravilhosos e inesquec�veis momentos ao lado das pessoas que amo. Nossa Senhora, Maria, m�e de Jesus, por ter passado na frente em todos os momentos, sempre a me proteger.

Agrade�o e dedico este trabalho aos meus pais (Marcelo e Socorro de Sales), � minha futura esposa Juliana Assun��o ``de Sales'' e ao nosso filho(a) que estamos esperando. Aos meus ador�veis pais que em inumerosos momentos souberam me mostrar os caminhos a seguir, educando-me e ajudando-me incondicionalmente a alcan�ar os meus objetivos, dando-me for�as sempre que eu precisei, unidos firmes j� conquistamos v�rias vit�rias. Por terem tanto amor, carinho e dedica��o, muit�ssimo obrigado por tudo. � Juliana, pelo amor e carinho que sempre me concedeu. Por in�meras vezes me mostrar que determinadas situa��es da vida podem ser resolvidas de maneira muito mais simples do que pensamos. Obrigado por compreender as minhas aus�ncias e stress durante toda essa jornada. Voc� faz muita diferen�a na minha vida, nunca esque�a disso. Agrade�o tamb�m a seus am�veis pais (Paulo e C�lia Assun��o), que incondicionalmente me apoiaram durante mais esta fase que est� chegando ao fim. 

Aos meus maravilhosos irm�os por serem para mim uma grande prova que a uni�o entre as pessoas pode efetivamente existir, independentemente da dist�ncia ou diferen�as. A todos os outros membros da fam�lia, meus tios, primos e os ``agregados'' que, embora n�o tenham sido mencionados (� muita gente!), sempre ter�o grande import�ncia em minha vida.

Aos meus orientadores Angelo Perkusich e Hyggo Oliveira de Almeida pela dedica��o conselhos e compreens�o.

Certamente sem todas essas pessoas este trabalho n�o teria acontecido. Por fim, agrade�o aos meus amigos de trabalho e de laborat�rio, em especial Elthon Alex, Evandro Costa, Fred Bublitz, Ig Ibert, Kyller Gorg�nio, Leandro Silva, M�rcio Ribeiro, Marcos Braga, Rafael Amorim, Rodrigo Peixoto, Thiago Ribeiro, Wendell Soares.